%
% File report.tex
%
%% Based on the style files for ACL 2018 and NAACL 2018, which were
%% Based on the style files for ACL-2015, with some improvements
%%  taken from the NAACL-2016 style
%% Based on the style files for ACL-2014, which were, in turn,
%% based on ACL-2013, ACL-2012, ACL-2011, ACL-2010, ACL-IJCNLP-2009,
%% EACL-2009, IJCNLP-2008...
%% Based on the style files for EACL 2006 by 
%%e.agirre@ehu.es or Sergi.Balari@uab.es
%% and that of ACL 08 by Joakim Nivre and Noah Smith

\documentclass[11pt,a4paper]{article}
\usepackage[hyperref]{naaclhlt2019}
\usepackage{times}
\usepackage{latexsym}
\usepackage{todonotes}
\usepackage{url}
\usepackage{units}

\aclfinalcopy % Uncomment this line for the final submission
%\def\aclpaperid{***} %  Enter the acl Paper ID here

%\setlength\titlebox{5cm}
% You can expand the titlebox if you need extra space
% to show all the authors. Please do not make the titlebox
% smaller than 5cm (the original size); we will check this
% in the camera-ready version and ask you to change it back.

\newcommand\BibTeX{B{\sc ib}\TeX}

\title{One Million Posts Corpus}

\author{Jens Becker \and Julius Plehn \and Oliver Pola \\ 
	Language Technology Group \\
	Fachbereich Informatik \\
	Fakultät für Mathematik, Informatik und Naturwissenschaften \\
	Universität Hamburg
}

\date{28.02.2020}

\begin{document}
\maketitle
\begin{abstract}
The purpose of this term paper is to summarize the development effort put into the creation of a multi-model neural network for category classification. This contains used preprocessing steps of the initial corpus, integration of an embedding layer and the actual deep learning network. Finally, an evaluation is given as well as ideas for future work. \todo{we conclude that...}

 
\end{abstract}

\section{Introduction}

...~\cite{Schabus17, Schabus18}


\section{Corpus}
The corpus used stems from the Austrian newspaper \textit{DER STANDARD} which contains posts that are labelled according to nine distinct categories and is available at \url{https://ofai.github.io/million-post-corpus/}. The posts have been written during a time period from 2015-2016 and contain 1 million unlabelled posts as well as 11.773 hand-labelled posts.


\subsection{Categories}
In Table~\ref{tab:categories} an overview of included categories is shown. For most categories 3599 labelled posts are available. This first column means that for example of those 11.773 hand-labelled posts 3599 posts where analysed for the \textit{Off Topic} category. The second column shows that in this case 580 posts where indeed \textit{Off Topic}, resulting in 16\% of the labelled posts of this specific category. As we want to build a model that is capable of identifying multiple categories we use only posts that are annotated as 0 or 1 for each category. This can result in unbalanced datasets where for example the \textit{Personal Stories} category is only annotated positively in 1\% of the posts. 
\begin{table*}[t!]
	\centering
	\begin{tabular}{l r r r r r}
		& Labeled & \multicolumn{2}{c}{Does apply} & \multicolumn{2}{c}{We apply} \\
		\hline
		Sentiment Negative & 3599 & 1691 & 47\% \\
		Sentiment Neutral & 3599 & 1865 & 52\% \\
		Sentiment Positive & 3599 & 43 & 1\% \\
		Off Topic & 3599 & 580 & 16\% \\
		Inappropriate & 3599 & 303 & 8\%\\
		Discriminating & 3599 & 282 & 8\%\\
		Possibly Feedback & 6038 & 1301 & 22\% & 72 & 2\%\\
		Personal Stories & 9336 & 1625 & 17\% & 47 & 1\%\\
		Arguments Used & 3599 & 1022 & 28\%\\
	\end{tabular}
	\caption{Categories of posts and their distribution according to \cite{Schabus17}}
	\label{tab:categories}
\end{table*}


%\begin{description}
%	\item[SentimentNegative] ...
%	\item[SentimentNeutral] ...
%	\item[SentimentPositive] ...
%	\item[OffTopic] ...
%	\item[Inappropriate] ...
%	\item[Discriminating] ...
%	\item[PossiblyFeedback] ...
%	\item[PersonalStories] ...
%	\item[ArgumentsUsed] ...
%\end{description}

\section{NLP}

...


\subsection{Word2Vec}

...%\cite{?}


\section{DeepLearning}

...


\subsection{Model}
\begin{figure}
	\centering
	\includegraphics[trim={0cm 19.5cm 0cm 3cm},clip,page=2, width=0.8\textwidth]{img/model}
	\caption{Applied deep learning network}
	\label{fig:model}
\end{figure}


\subsection{Training}

...


\section{Results}

...


\subsection{Single-Category}

...


\subsection{Multi-Category}

...

\section{Future Work}


\bibliography{references}
\bibliographystyle{acl_natbib}
\appendix

\section{Appendices}
\label{sec:appendix}





\end{document}
